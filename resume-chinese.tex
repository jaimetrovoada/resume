 %& -job-name=jaimetrovoada-resume-chinese
\documentclass[a4paper, 20pt]{article}

\title{Resume (chinese version)}
\author{Jaime Trovoada Dias Lourenco}
\date{July 2022}

\usepackage{latexsym}
\usepackage[empty]{fullpage}
\usepackage{titlesec}
\usepackage{marvosym}
\usepackage[usenames,dvipsnames]{color}
\usepackage{verbatim}
\usepackage{enumitem}
\usepackage[pdftex]{hyperref}
\usepackage{fancyhdr}
\usepackage{CJKutf8}
\usepackage{xcolor}

\pagestyle{fancy}
\fancyhf{} % clear all header and footer fields
\fancyfoot{}
\renewcommand{\headrulewidth}{0pt}
\renewcommand{\footrulewidth}{0pt}
\renewcommand{\familydefault}{\sfdefault}

% Adjust margins
\addtolength{\oddsidemargin}{-0.530in}
\addtolength{\evensidemargin}{-0.375in}
\addtolength{\textwidth}{1in}
\addtolength{\topmargin}{-.45in}
\addtolength{\textheight}{1in}

\urlstyle{rm}

\raggedbottom{}
\raggedright{}
\setlength{\tabcolsep}{0in}

% Sections formatting
\titleformat{\section}{
  \vspace{-10pt}\scshape\raggedright\large
}{}{0em}{}[\color{black}\titlerule{}\vspace{-6pt}]{}

%-------------------------
% Custom commands
\newcommand{\resumeItem}[2]{
  \item\small{
    \textbf{#1}{: #2 \vspace{-2pt}}
  }
}

\newcommand{\resumeItemWithoutTitle}[1]{
  \item\small{
    {\vspace{-2pt}}
  }
}

\newcommand{\resumeSubheading}[4]{
  \vspace{-1pt}\item
    \begin{tabular*}{0.97\textwidth}{l@{\extracolsep{\fill}}r}
      \textbf{#1} & #2 \\
      #3 & #4 \\
    \end{tabular*}\vspace{-5pt}
}


\newcommand{\resumeSubItem}[2]{\resumeItem{#1}{#2}\vspace{-3pt}}

\renewcommand{\labelitemii}{$\circ$}

\newcommand{\resumeSubHeadingListStart}{\begin{itemize}[leftmargin=*]}
\newcommand{\resumeSubHeadingListEnd}{\end{itemize}}
\newcommand{\resumeItemListStart}{\begin{itemize}}
\newcommand{\resumeItemListEnd}{\end{itemize}\vspace{-5pt}}

%-----------------------------
%%%%%%  CV STARTS HERE  %%%%%%

\begin{document}
\begin{CJK*}{UTF8}{gbsn}

%----------HEADING-----------------
\begin{tabular*}{\textwidth}{l@{\extracolsep{\fill}}r}
  \textbf{{\LARGE Jaime Trovoada}}\\
  \href{https://jaimetrovoada.vercel.app/}{个人网站: jaimetrovoada.vercel.app}  & 邮箱: \href{mailto:jaimetrovoada@gmail.com}{jaimetrovoada@gmail.com}\\
  \href{https://github.com/jaimetrovoada}{Github: jaimetrovoada} & 手机号: \href{tel:+8618202617004}{+86 182 0261 7004}\\
  \href{https://www.linkedin.com/in/jaimetrovoada630/}{LinkedIn: linkedin.com/in/jaimetrovoada630} & \textit{Tianjin, China}\\
\end{tabular*}


%-----------EXPERIENCE-----------------
\vspace{5pt}
\section{\textcolor[HTML]{E36C09}{\textbf{工作经验}}}
  \resumeSubHeadingListStart{}
    \resumeSubheading{Frontend Developer (Web 前端)}{北京}
    {北京燃星科技有限公司}{08月2021 - 08月2022}
    \resumeItemListStart{}
      \item {使用Storybook创建可重用且易于维护的组件。}
      \item {与开发人员和设计师团队合作创建一致且用户友好的设计系统。}
      \item {具备学习新技术和适应变化的能力。}
      \item {学会了如何实现Web3应用程序。}
    \resumeItemListEnd{}
  \resumeSubHeadingListEnd{}


%-----------EDUCATION-----------------
\vspace{5pt}
\section{\textcolor[HTML]{E36C09}{\textbf{教育}}}
  \resumeSubHeadingListStart{}
    \resumeSubheading{天津大学}{天津}
      {本科, 计算机科学与技术 }{09月2017 - 07月2023}
    \resumeSubheading{天津大学}{天津}
      {汉语}{02月2017 - 07月2017}
    \resumeSubheading{辅仁大学}{台湾}
      {汉语}{09月2016 - 02月2017}
    \resumeSubheading{Coursera, Johns Hopkins University}{coursera.com}
      {HTML, CSS, and JavaScript for Web Developers}{09月2020 - 10月2020}
  \resumeSubHeadingListEnd{}
	    
%-----------CERTS-----------------
\vspace{5pt}
\section{\textcolor[HTML]{E36C09}{\textbf{认证}}}
\resumeSubHeadingListStart{}
\resumeSubheading{University of Helsinki - MOOC}{fullstackopen.com}
{Full Stack Development}{2023}
\resumeSubHeadingListEnd{}

%-----------SKILLS-----------------
\vspace{5pt}
\section{\textcolor[HTML]{E36C09}{\textbf{技术和技能}}}
	\resumeSubHeadingListStart{}
	  \resumeSubItem{技术}{\quad JavaScript/TypeScript, React, NextJS, TailwindCSS, Bootstrap, Vue, HTML/CSS/SCSS, Git, Go(lang), NodeJS, Docker, MongoDB}
	  \resumeSubItem{语言}{\quad 葡萄牙语, 英文, 中文 (HSK4)}
  \resumeSubHeadingListEnd{}


%-----------PROJECTS-----------------
\vspace{5pt}
\section{\textcolor[HTML]{E36C09}{\textbf{项目}} \tiny{(更多在我的 GitHub 上)}}
  \resumeSubHeadingListStart{}
    \resumeSubItem{Patent Inquiry System [React, NodeJS, Express, and MySQL]}
      {实现了一个连接到 MySQL 数据库的后端和一个包含与专利相关的信息的服务的前端,并允许用户搜索其中包含的信息}
      \vspace{2pt}
      \resumeSubItem{GruvIt [React + TypeScript, Go, Docker]}
      {开发了一个服务,在 \href{https://github.com/paulopacitti/gruvbox-factory}{gruvbox-factory} python 包的帮助下将上传的图片修改为 \href{https://github.com/morhetz/gruvbox}{gruvbox} 调色板,然后返回给用户。}
      \begin{itemize}
        \item{使用 Go 和 \href{https://github.com/gin-gonic/gin}{Gin}框架构建的后端,使编写 API 更容易,减少 50\% 的设置时间,并受益于 Go 的先天性能;}
        \item{上传文件的 POST 路由。 然后,通过 gruvbox-factory 工具修改文件;}
        \item{GET 路由用于检索最终文件;}
      \end{itemize}
      \vspace{2pt}
    \resumeSubItem{Portfolio Template Website [React + TypeScript]}
      {发布了一个 GitHub 存储库模板,用于制作个人网站并使用 GitHub Pages 进行部署;}
      \vspace{2pt}
    \resumeSubItem{To-Do WebApp [HTML/CSS, JavaScript]}
      {开发了一个使用浏览器 localStorage API 来保存数据的待办事项应用程序。}
      \vspace{2pt}
    \resumeSubItem{Weather WebApp [React]}
      {编写了一个使用浏览器位置 API 和 OpenWeatherMap API 的简单天气应用程序。 它可以显示当前和次日的天气信息。}
      \vspace{2pt}
  \resumeSubHeadingListEnd{}

\clearpage\end{CJK*}
\end{document}
